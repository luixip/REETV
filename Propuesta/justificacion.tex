En concordancia con el Plan de Desarrollo Profesoral \parencite{UQPlanDesarrolloPro2016},  y el Proyecto Educativo de la Facultad de Ingeniería \parencite{UQProyectoEducativoFacultad}, el programa de Ingeniería de Sistemas y Computación de la Universidad del Quindío, en el primer semestre de 2017 definió una reforma curricular \parencite{UQACA075}, en la cual se destaca la creación del área de profundización denominada Infraestructura de Tecnología Informática (TI), la cual comprende entre otros espacios académicos los siguientes: \textit{Infraestructura Computacional}, \textit{Administración de Infraestructura}, \textit{Gestión de Servicios de TI } y \textit{Computación en la Nube}; estos espacios académicos tienen relación con las tecnologías de virtualización e incluso algunos lo han declarado explícitamente en su \textit{syllabus}. Por esta razón el presente trabajo genera un apoyo directo al ejercicio docente de estos espacios académicos, al igual que se esperan beneficios indirectos a la demás comunidad académica de la Facultad de Ingeniería en la Universidad del Quindío y demás instituciones de educación con necesidades similares.\\

El conocimiento y utilización de las tecnologías de virtualización en instituciones de educación superior y particularmente en la Universidad del Quindío, puede considerarse como una acción estratégica en el sentido de que se pueden aprovechar al máximo los recursos computacionales existentes \parencite{Klement2017}, permitiendo la generación de escenarios para la construcción de laboratorios virtuales en diversos espacios académicos, estos laboratorios generan un alto valor académico sin incurrir en nuevas inversiones en TI.
\\

Para cualquier comunidad académica es de gran importancia el uso de las tecnologías de virtualización, debido a que estas tecnologías tienen un impacto positivo en el medio ambiente, incluso se asocia al concepto de Green Computing \parencite{Thathera2015}, esto se sustenta en que, a través de estas tecnologías, es posible la utilización adecuada de la infraestructura de TI y la reducción en la adquisición de nuevo hardware que por lo general trabaja con tasas bajas de carga computacional. Otra razón para la utilización de las tecnologías de virtualización se centra en la reducción del costo total de propiedad de la infraestructura de TI, representado en la disminución de consumo eléctrico consecuencia de una menor cantidad de hardware que necesita permanecer encendido para sostener los procesos computacionales en las organizaciones. \\

Las tecnologías de virtualización también ayudan en la reducción de la complejidad de los sistemas informáticos, debido a que las aplicaciones pueden estar en entornos aislados a nivel lógico aun cuando se ejecuten en un mismo hardware \parencite{Chiueh2005}, esto significa que las aplicaciones pueden tener un sistema operativo de forma exclusiva para evitar conflictos con otras aplicaciones. \\

Basados en la abstracción del hardware que realizan las tecnologías de virtualización \parencite{Chiueh2005}, y a modo de ejemplo, es posible suponer que si se construye un laboratorio virtual, este no dependa de un hardware específico, por el contrario, este laboratorio podrá ser movido a otro hardware de una forma fácil, lo que ayuda en los procesos de continuidad en la prestación de servicios en diversos ámbitos. \\

Aunque existe suficiente bibliografía que puede ser utilizada por profesores y estudiantes para abordar directamente las tecnologías de virtualización, el proceso de selección de los ejemplos o adaptación de ellos en casos prácticos, no es una tarea trivial. En ocasiones los ejemplos están situados en supuestos muy abstractos que no permiten concretar la apropiación de conceptos o están plagados de inferencias que hacen aún más complejo el proceso de aprendizaje de esta temática. \\

Los recursos educativos propuestos en este trabajo pueden contribuir notablemente en el conocimiento y apropiación de las tecnologías de virtualización por parte de los profesores y estudiantes del programa de Ingeniería de Sistemas y Computación, porque aportan conocimiento de mucha actualidad y alto impacto en el sector productivo de la ingeniería.


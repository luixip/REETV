En los últimos años las tecnologías de virtualización han sido utilizadas ampliamente por diversas organizaciones para cubrir sus necesidades de los sistemas y aplicaciones, lo anterior, gracias a los beneficios que presentan las tecnologías de virtualización con relación a la partición o consolidación de los recursos informáticos, aislamiento de ambientes operativos, búsqueda de condiciones particulares de seguridad y facilidad en la administración y soporte de la infraestructura de tecnologías de información, entre otros aspectos \parencite{Pessolani2012}.\\

Algunas de las razones para la implementación de tecnologías de virtualización en las organizaciones tienen están relacionadas directamente con el aspecto económico, lo anterior debido a que, al combinar la consolidación de hardware con la virtualización, es posible obtener mejoras en los valores relacionados con el retorno de las inversiones (ROI por las siglas en inglés \textit{Return Of Invesment}) y reducciones en el costo total de propiedad (TCO por las siglas en inglés \textit{Total Cost Ownership}).\\

Particularmente, en la última década las tecnologías de virtualización han terminado de emerger y se han logrado consolidar en el mundo de las tecnologías de la información \parencite{Kampert2010}. Esta tendencia ha logrado que muchas organizaciones inicien su propio desarrollo de tecnologías de virtualización, dando lugar a su vez a una gran diferencia en los tipos de tecnologías de virtualización desarrolladas. Parte de esta situación se refleja en el hecho de que en la documentación disponible acerca de las tecnologías de virtualización abundan los documentos sobre de procedimientos técnicos, pero ciertamente existe menos publicaciones acerca de la clasificación de los tipos de virtualización, lo que hace difícil la unificación de criterios sobre la denominación y las fronteras conceptuales entre las diversas tecnologías de virtualización existentes y emergentes. \\

Por lo anteriormente expuesto, y antes de hacer referencia a los aspectos técnicos propios de las tecnologías de virtualización, es relevante dar claridad a los conceptos relacionados con esta temática tal como se detalla en la siguiente sección. \\

\section{Ámbitos de aplicación de las tecnologías de virtualización} \label{ambitos}

Las tecnologías de virtualización brindan una variedad de beneficios

que pueden ser aprovechados en diversos ámbitos como la industria, la academia y la investigación. Cada uno de estos ámbitos suele aprovechar los beneficios que brindan las tecnologías de virtuales. Algunos de los principales beneficios se describen a continuación:\\

\begin{itemize}
	\item 
\end{itemize}

  de aplicación debido debido a los beneficios que presentan respecto a menor TCO y mejor ROI. 



\subsection{Ámbito Industrial}

\begin{description}
	\item[Centro de procesamiento de datos] Empresas especializadas en infraestructura con centro de procesaminto de datos (CDP o en inglés \textit{Datacenter}).
\end{description}



Facilita la aplicación de los planes de continuidad de negocio en las organizaciones. 

Desarrollo de software

Favorece el rápido despliegue de ambientes de trabajo, reduciendo así el tiempo de entrada en producción de nuevos empleados en las organizaciones. \\


Internet de las Cosas o IoT (por las siglas en inglés Internet Of Things) es una área que se está logrando potenciar gracias a la virtualización basada en el sistema operativo, también llamada \textit{Virtualización Liviana} (en inglés Lightwiht Virtualization). 



\subsection{Académico}

Enseñanza de las tecnologías de virtualización


Virtualización de entornos de trabajo (docker)


Lenguajes de programación (Java Virtual Machine y .NET)

\section{Identificación de herramientas de virtualización} \label{sec:herramientas}
\vspace{5mm}
% Objetivo específico: 
% dentificar y describir herramientas que permitan implementar las tecnologías de virtualización. 

% Temática a desarrollar. Identificación de herramientas de virtualización:
% esta sección pretende describir algunas de las herramientas de las tecnologías de virtualización existentes, reconociendo sus características funcionales. 

Con se indicó en las secciones anteriores, la virtualización tiene diversas formar de llevarse a cabo y puede ser aplicada a distintos ámbitos. Considerando lo anterior, es evidente que en la actualidad existan también, una variedad considerable de herramientas tecnológicas que implementan los distintos tipos de virtualización.  Por lo tanto,  en esta sección se describen algunas de las herramientas de las tecnologías de virtualización, señalando también sus características funcionales.\\

Para la presentación de las tecnologías de virtualización, se utiliza el modelo taxonómico propuesto por \textcite{Pessolani2012} (Ver apartado \ref{sec:taxonomía}). Teniendo presente aspectos considerados en la propuesta de este trabajo (ver apartado \ref{Objetivos}), se consideran tres categorías taxonómicas que  corresponden a \textit{virtualización de hardware o sistema}, \textit{Para-virtualización} y \textit{virtualización basada en sistema operativo}.\\

\subsection{Herramientas de Virtualización de Hardware o Sistema}
\vspace{5mm}

- VMware \\
- Virtual Box \\


\subsubsection{h1}
\vspace{5mm}
\subsubsection{h2}
\vspace{5mm}

\subsection{Herramientas de Para-Virtualización}
\vspace{5mm}

- Xen\\
- Denali\\
- Hyper-V\\



\subsubsection{h1}
\vspace{5mm}
\subsubsection{h2}
\vspace{5mm}

\subsection{Herramientas de virtualización basada en Sistema Operativo}
\vspace{5mm}
\subsubsection{h1}
\vspace{5mm}
\subsubsection{h2}
\vspace{5mm}


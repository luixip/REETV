En esta sección se presentan algunos de los fundamentos conceptuales acerca de las tecnologías de virtualización, los cuales permiten brindar claridad en la terminología para establecer un marco común de contextualización utilizado en el presente trabajo. 

\section{Virtualización}

La virtualización puede ser considerada como una tecnología, que ofrecen a las organizaciones una gran variedad de ventajas. Técnicamente y según lo señala Kusnetzky (2011), esta tecnologías consiste en abstraer aplicaciones y los componentes subyacentes del hardware que los soporta para presentar una vista lógica o virtual de estos recursos. Esta vista lógica puede en ocasiones ser notablemente diferente a la vista física; por lo general, la vista lógica o virtual se construye a partir del exceso de recursos computacionales tales como el poder de procesamiento, memoria, capacidad de almacenamiento o incluso ancho de banda (Stallings, 2015). Cuando la vista lógica se utiliza para la combinación de recursos computacionales, estos pueden ser presentados como uno o varios entornos operativos. 
\\\\
Por ejemplo, si se tienen varios computadores físicos, estos pueden ser vistos como un único gran recurso (agregación de recursos) con características superiores a los elementos que lo integran; otro ejemplo consiste en tener un solo computador físico, el cual puede ser visto como múltiples instancias con características inferiores que el computador físico (división de recursos). 
\\\\
Todas estas capacidades de la virtualización pueden tener lugar usando metodologías como partición o agregación de hardware y software, simulación de máquina parcial o completa, emulación y tiempo compartido, etc.  (Chiueh, 2005).
\\\\
Una de las formas más comunes de usar la virtualización es mediante la división de recursos, en este caso, es necesario agregar una capa de virtualización sobre el recurso real, tal como se muestra en la Ilustración 1. Esta capa de virtualización permite que varios recursos virtuales sean ejecutados simultáneamente ubicados sobre un mismo recurso real.


\section{Máquina real}

\section{Máquina virtual}

\section{Monitor de Máquinas Virtuales}

\section{Taxonomía de Tecnologías de Virtualización}

\subsubsection{Virtualización de Hardware o Sistema}
\subsubsection{Para-Virtualización}
\subsubsection{Virtualización basada en Sistema Operativo}
\subsubsection{Virtualización a nivel de proceso o Aplicación}
\subsubsection{Virtualización de Sistema Operativo}

\section{Contexto histórico de las tecnologías de virtualización}

\section{Ámbitos de aplicación de las tecnologías de virtualización}

\chapter{Herramientas de Virtualización}

\section{Virtualización de Hardware o Sistema}
\subsection{h1}
\subsection{h2}
\subsection{h3}

\section{Para-Virtualización}
\subsection{h1}
\subsection{h2}
\subsection{h3}

\section{Virtualización basada en Sistema Operativo}
\subsection{h1}
\subsection{h2}
\subsection{h3}
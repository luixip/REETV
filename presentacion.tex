%\chapter*{Presentación}
%\markboth{Presentacion}{Presentacion}

{\huge \textbf{Presentación}}\\

Este documento corresponde al trabajo presentado como requisito para el ascenso en el escalafón docente de la categoría de \textit{Profesor Asociado} en la Universidad del Quindío. Tal como se establece en el Artículo 4o del Acuerdo del Consejo Superior número 010 del 4 de marzo de 1996 \parencite{UQ1996}.\\

La propuesta presentada, se centra en las tecnologías de virtualización y busca  realizar una revisión bibliográfica que permita la descripción de fundamentos, contexto histórico y ámbitos de aplicación de estas tecnologías. Todo esto para la construcción de un conjunto de recursos educativos orientados a la enseñanza de las tecnologías de virtualización, buscando un aporte significativo a la docencia, especialmente en los espacios académicos del área de infraestructura de Tecnología Informática (TI) del programa de Ingeniería de Sistemas y Computación en la Facultad de Ingeniería en la Universidad del Quindío.\\

La virtualización según lo señala \textcite{Goldworm2007} y \textcite{Kusnetzky2011} es una mezcla de tecnologías que permiten realizar una abstracción del hardware presentándolo como una vista lógica del mismo, lo que permite que múltiples “máquinas virtuales” se ejecuten en una misma máquina física ( o real). Esta situación trae consigo un esquema de trabajo con mayor flexibilidad, que a su vez facilita la adaptación de la infraestructura de TI hacia las necesidades de las organizaciones. El concepto de tecnologías de virtualización utilizado en esta propuesta, está suscrito en la taxonomía descrita por \textcite{Pessolani2012}, que a su vez cumple los requisitos asociados a la virtualización concebidos por \textcite{Popek1974}. Partiendo de dicha taxonomía, esta propuesta se basa en los siguientes aspectos: a) Virtualización de hardware, b) Para-virtualización, y c) Virtualización basada en sistema operativo. La razón de los aspectos a considerar tiene relación con los contenidos programáticos de los espacios académicos del área de infraestructura de TI del Programa Ingeniería de Sistemas y Computación.\\

Cabe resaltar la actualidad y el impacto de la temática a tratar en la presente propuesta, toda vez que gran parte de las organizaciones hoy en día utilizan la tecnología computacional como un elemento fundamental para potenciar el cumplimiento de sus objetivos \parencite{Pessolani2012}, por lo que se hace necesario contar con el conocimiento preciso acerca de la especificación dinámica de recursos computacionales utilizando tecnologías de virtualización \parencite{Hui2014}. También es importante reconocer y saber gestionar las tecnologías de virtualización que hacen posible soportar el despliegue de recursos tecnológicos en soluciones empresariales de bajo, mediano y alto impacto \parencite{Chiueh2005}. \\

El trabajo presentado consta de dos partes, la Parte \ref{Parte1}, corresponde a la descripción general de la propuesta presentada ante el Consejo de la Facultad de Ingeniería en la Universidad del Quindío. A su vez, esta parte comprende las siguientes secciones: \ref{justificacion}) Justificación, \ref{Objetivos}) Objetivos y \ref{tematicaADesarrollar}) Temática a desarrollar. Es importante dar claridad al lector que los objetivos y alcance de la propuesta fueron aprobados por el Consejo de la Facultad de Ingeniería, mediante el acta No. 16 del 16 de agosto de 2018. La Parte \ref{Parte2}, corresponde al trabajo de ascenso desarrollado, el cual incluye secciones como  \ref{Introduccion}) Introducción, el  \ref{marcoConceptual}) marco conceptual que a su vez comprende \ref{fundamentos}) fundamentos, XXX) contexto histórico y XXX) ámbitos de aplicación, identificación de herramientas de virtualización y recursos educativos acerca de las tecnologías de virtualización entre otras.\\